
\documentclass[a4paper]{article}

\usepackage[margin=2.5cm]{geometry}
\usepackage{graphicx}
\usepackage{listings}
\usepackage{color}

\lstset{
  language=C,
  basicstyle=\small,
  numbers=none,
  numberstyle=\footnotesize,
  stepnumber=1,
  numbersep=5pt,
  backgroundcolor=\color{white},
  showspaces=false,
  showstringspaces=false,
  showtabs=false,
  frame=single,
  tabsize=2,
  captionpos=b,
  breaklines=true,
  breakatwhitespace=false,
  escapeinside={\%*}{*)}
}



\newcommand{\opset}{{\em op\_set }}
\newcommand{\opmap}{{\em op\_map }}
\newcommand{\opdat}{{\em op\_dat }}
\newcommand{\opsetN}{{\em op\_set}} % no space version
\newcommand{\opmapN}{{\em op\_map}} % no space version
\newcommand{\opdatN}{{\em op\_dat}} % no space version

\newcommand{\opdeclset}{{\em op\_decl\_set }}
\newcommand{\opdeclmap}{{\em op\_decl\_map }}
\newcommand{\opdecldat}{{\em op\_decl\_dat }}
\newcommand{\opdeclgbl}{{\em op\_decl\_gbl }}
\newcommand{\opdeclsetN}{{\em op\_decl\_set}} % no space version
\newcommand{\opdeclmapN}{{\em op\_decl\_map}} % no space version
\newcommand{\opdecldatN}{{\em op\_decl\_dat}} % no space version
\newcommand{\opdeclgblN}{{\em op\_decl\_gbl}} % no space version


\newcommand{\opputvalue}{{\em op\_put }}
\newcommand{\opyield}{{\em op\_yield }}
\newcommand{\opputvalueN}{{\em op\_put}} % no space version
\newcommand{\opyieldN}{{\em op\_yield}} % no space version

\newcommand{\hydloop}{hyd\_par\_loop }
\newcommand{\hydloops}{hyd\_par\_loops }
\newcommand{\hydloopN}{hyd\_par\_loop} % no space version
\newcommand{\hydloopsN}{hyd\_par\_loops} % no space version


\newcommand{\oploop}{op\_par\_loop }
\newcommand{\oploops}{op\_par\_loops }
\newcommand{\oploopN}{op\_par\_loop} % no space version
\newcommand{\oploopsN}{op\_par\_loops} % no space version

\title{OP2 Common C/C++ and Fortran Library \\ Implementation Design Description}
\author{Carlo Bertolli and Adam Betts\\ Imperial College London}

\begin{document}
\maketitle

\section{Introduction}
This document gives an overview of the OP2 library designed to be
used by both C/C++ and Fortran applications. The library has been
re-designed starting from C++ and Fortran versions, and its
core functions are implemented in C99.


\section{Directory Organisation}
The main OP2 Common directory contains the following subdirectories:

\begin{itemize}

\item {\bf op2}: this directory contains the C/C++ and Fortran OP2
  support (see below).

\item {\bf apps}: this directory contains the example applications for
  C++ and Fortran.

\item {\bf doc}: this directory contains the generic OP2 Common
  documentation, as this document.

\item {\bf translator}: this directory contains the OP2
  source-2-source compilers. It is currently empty, except for the
  MATLAB compilers, as the translators are developed in different
  repositories.

\end{itemize}

\section{The OP2 Directory}

As specified above, the OP2 directory contains the OP2 support code
needed by Fortran and C/C++. The support files are characterised in
C/C++ files and in Fortran ones. While the first ones are uniquely
implemented in C/C++ and they are self-contained, the Fortran support
makes use of the C/C++ one and it is programmed both in Fortran and C.

\subsection{The OP2 C Support}
The main subdirectories of ``op2/c'' are the source one (``src'') and
the include one (``include''). Further directories are ``doc'' which
contains C/C++ OP2 documentation, ``lib'' and ``obj'' that are used by
the Makefile for the compilation results.

The ``src'' directory contains:

\begin{itemize}

\item The ``op\_lib\_core.cpp'' file, which implements the core library
  routines used to initialise the OP2 data structures, like \opset,
  \opmap, and \opdatN. It also defines the lists of actually declared
  OP2 variables, namely OP\_set\_list, OP\_map\_list and OP\_dat\_list,
  and related information. The functions provided in this file are
  called by upper layer functions and they should not be called
  directly from user programs.

\item The ``op\_reference.c'' file provides a set of wrappers to core
  library functions, to be used in reference implementation. This file
  does {\it not} contain the reference implementation of \oploop
  functions.

\item The ``op\_rt\_support.c'' file implements the OP2 run-time support
  functions used by any OP2 back-end (i.e. C/C++ and Fortran) that
  actually needs them. It provides an implementation of the
  ``op\_plan\_core'' function, which builds an execution plan (i.e.
  based on partitioning and colouring an OP2 set) based on the input
  data to an indirect OP2 parallel loop. It also defines the OP2 plan
  list variable, called ``OP\_plans''. It is worth of noticing that
  this files also provides an op\_rt\_exit function which should be
  called by any OP2 backend using these run-time functions when
  terminating the OP2 library, via the high-level op\_exit.

\item A ``cuda'' directory containing an implementation of high-level
  OP2 functions properly extending the implementation of the core
  library. The ``op\_cuda\_decl.c'' file provides an implementation of
  the data structure OP2 declaration functions, which is alternative
  to the reference one and it must be used by C/C++ and Fortran CUDA
  back-ends. In particular, the \opdecldat function first calls the
  op\_decl\_dat\_core function and then it copies the declared data
  from host to device memory space. The ``op\_cuda\_rt\_support.c'' file
  provides an implementation of utility functions wrapping CUDA
  functions (e.g. cudaMalloc), and a wrapper for the op\_plan\_core
  function, which copies part of the generated plan information from
  host to device memory space. Both files are used by the Fortran CUDA
  back-end.

\item An ``openmp'' directory contaning the
  ``op\_openmp\_decl.c'' file, which wraps core library functions.
  This file is used also by the Fortran OpenMP back-end.

\end{itemize}


The ``include'' directory contains the following header files:

\begin{itemize}

\item op\_lib\_core.h: this file provides the declaration of the OP2
  core library functions (see the corresponding implementation file
  above).

\item op\_lib\_c.h: this file provide the declaration of the
  high-level OP2 functions in C. It is included by one of the Fortran
  wrappers.

\item op\_lib\_cpp.h: this file extends the previous one with C++ OP2
  high-level functions, related to the declaration of \opdat variables
  (using templates), constants and global arguments to \oploopN.
  The file directly contains the definition of the function, which
  call the functions declared in the previous header by properly
  transforming the templated calls to plan C calls.

\item op\_openmp\_rt\_support.h: this file declares the op\_plan\_get
  function wrapper, which is used by the OpenMP back-ends.

\item op\_cuda\_rt\_support.h: this file declares the functions
  defined in the op\_cuda\_rt\_support.c file (see above).

\item op\_cuda\_reduction.h: this file provides a templated
  implementation of the op\_reduction routines for CUDA back-ends. It
  is not part of the op\_cuda\_rt\_support.h file (as it was in
  previous versions) because the previous file is included by a source
  file that is compiled with a plain C compiler, i.e. not supporting
  templates. This design is a consequence of the choice of avoiding to
  label C functions using the notation ``extern C'' to be called by
  both Fortran and C++.

\end{itemize}

It is to be remarked that the C++ and Fortran support might compile
the same files with different compilers (e.g. nvcc and gcc). For this
reason, some of the include file directly include low level CUDA
headers like:

\begin{figure}[h!]
 \centering
\begin{lstlisting}
#include <cuda.h>
#include <cuda_runtime.h>
#include <cuda_runtime_api.h>
#include <device_launch_parameters.h>
#include <device_functions.h>
\end{lstlisting}
\caption{Example of declaration of op\_set variables.}
 \label{fig:op-set-example}
\end{figure}



\subsubsection{Compiling OP2 C Support}

The ``op2/c'' directory currently contains a Makefile to build proper
libraries to be used by OP2 applications. The invocation of the
``make'' command in the directory will (hopefully) produce the
following libraries:

\begin{itemize}

\item {\bf libop2\_reference.a}: this is the library providing the
  reference implementation of OP2 calls. It depends on the core library
  functions and on the reference version-related source files (see
  above).

\item {\bf libop2\_rt\_support.a}: this is the OP2 run-time support
  library, and it includes a compilation of the op\_rt\_support.c file
  (see above), i.e. an implementation of the op\_plan\_core function
  and associated functions and variables.

\item {\bf libop2\_cuda.a}: this is the library providing an
  implementation of OP2 functions for CUDA back-ends.

\item {\bf libop2\_openmp.a}: this is the library providing an
  implementation of OP2 functions for OpenMP back-ends.

\end{itemize}

For example, to link an application against the CUDA library, it is sufficient to
specify in the g++ command line the ``-lop2\_cuda''  option (i.e.
there is no need of using the prefix ``lib'' and the extension ``.a''.

\section{The Applications Directory}

The application directory (``apps'') contains examples in Fortran and C++.

\subsection{The C+ Application Directory}
This directory contains the airfoil and jacobi examples. The airfoil
application makes use of the OP2 C common library. To compile it, it
is required to set the OP2 environment variable to the path of the OP2
Common ``op2'' directory (i.e. the one containing the C/C++ and
Fortran support directories). For CUDA, it is required to set the
CUDA\_INSTALL\_PATH variables to the path of the local CUDA
installation. The compilation proceeds in the following
way:

\begin{itemize}

\item The sequential implementation is compiled with g++ and linked
  against the op2\_reference library.

\item The OpenMP implementation is compiled with g++ and linked
  against the OP2 run-time library and the openmp one. The run-time
  library is needed because the OpenMP back-end makes use of the OP2
  function to build OP2 plans.

\item The CUDA implementation is compiled with nvcc and g++ and linked
  against the OP2 run-time library (again because the back-end uses
  the plan function) and the OP2 CUDA library.

\end{itemize}


\end{document}
